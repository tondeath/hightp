\documentclass[]{article}
\usepackage{lmodern}
\usepackage{amssymb,amsmath}
\usepackage{ifxetex,ifluatex}
\usepackage{fixltx2e} % provides \textsubscript
\ifnum 0\ifxetex 1\fi\ifluatex 1\fi=0 % if pdftex
  \usepackage[T1]{fontenc}
  \usepackage[utf8]{inputenc}
\else % if luatex or xelatex
  \ifxetex
    \usepackage{mathspec}
  \else
    \usepackage{fontspec}
  \fi
  \defaultfontfeatures{Ligatures=TeX,Scale=MatchLowercase}
\fi
% use upquote if available, for straight quotes in verbatim environments
\IfFileExists{upquote.sty}{\usepackage{upquote}}{}
% use microtype if available
\IfFileExists{microtype.sty}{%
\usepackage{microtype}
\UseMicrotypeSet[protrusion]{basicmath} % disable protrusion for tt fonts
}{}
\usepackage[margin=1in]{geometry}
\usepackage{hyperref}
\hypersetup{unicode=true,
            pdftitle={homework3},
            pdfborder={0 0 0},
            breaklinks=true}
\urlstyle{same}  % don't use monospace font for urls
\usepackage{color}
\usepackage{fancyvrb}
\newcommand{\VerbBar}{|}
\newcommand{\VERB}{\Verb[commandchars=\\\{\}]}
\DefineVerbatimEnvironment{Highlighting}{Verbatim}{commandchars=\\\{\}}
% Add ',fontsize=\small' for more characters per line
\usepackage{framed}
\definecolor{shadecolor}{RGB}{248,248,248}
\newenvironment{Shaded}{\begin{snugshade}}{\end{snugshade}}
\newcommand{\AlertTok}[1]{\textcolor[rgb]{0.94,0.16,0.16}{#1}}
\newcommand{\AnnotationTok}[1]{\textcolor[rgb]{0.56,0.35,0.01}{\textbf{\textit{#1}}}}
\newcommand{\AttributeTok}[1]{\textcolor[rgb]{0.77,0.63,0.00}{#1}}
\newcommand{\BaseNTok}[1]{\textcolor[rgb]{0.00,0.00,0.81}{#1}}
\newcommand{\BuiltInTok}[1]{#1}
\newcommand{\CharTok}[1]{\textcolor[rgb]{0.31,0.60,0.02}{#1}}
\newcommand{\CommentTok}[1]{\textcolor[rgb]{0.56,0.35,0.01}{\textit{#1}}}
\newcommand{\CommentVarTok}[1]{\textcolor[rgb]{0.56,0.35,0.01}{\textbf{\textit{#1}}}}
\newcommand{\ConstantTok}[1]{\textcolor[rgb]{0.00,0.00,0.00}{#1}}
\newcommand{\ControlFlowTok}[1]{\textcolor[rgb]{0.13,0.29,0.53}{\textbf{#1}}}
\newcommand{\DataTypeTok}[1]{\textcolor[rgb]{0.13,0.29,0.53}{#1}}
\newcommand{\DecValTok}[1]{\textcolor[rgb]{0.00,0.00,0.81}{#1}}
\newcommand{\DocumentationTok}[1]{\textcolor[rgb]{0.56,0.35,0.01}{\textbf{\textit{#1}}}}
\newcommand{\ErrorTok}[1]{\textcolor[rgb]{0.64,0.00,0.00}{\textbf{#1}}}
\newcommand{\ExtensionTok}[1]{#1}
\newcommand{\FloatTok}[1]{\textcolor[rgb]{0.00,0.00,0.81}{#1}}
\newcommand{\FunctionTok}[1]{\textcolor[rgb]{0.00,0.00,0.00}{#1}}
\newcommand{\ImportTok}[1]{#1}
\newcommand{\InformationTok}[1]{\textcolor[rgb]{0.56,0.35,0.01}{\textbf{\textit{#1}}}}
\newcommand{\KeywordTok}[1]{\textcolor[rgb]{0.13,0.29,0.53}{\textbf{#1}}}
\newcommand{\NormalTok}[1]{#1}
\newcommand{\OperatorTok}[1]{\textcolor[rgb]{0.81,0.36,0.00}{\textbf{#1}}}
\newcommand{\OtherTok}[1]{\textcolor[rgb]{0.56,0.35,0.01}{#1}}
\newcommand{\PreprocessorTok}[1]{\textcolor[rgb]{0.56,0.35,0.01}{\textit{#1}}}
\newcommand{\RegionMarkerTok}[1]{#1}
\newcommand{\SpecialCharTok}[1]{\textcolor[rgb]{0.00,0.00,0.00}{#1}}
\newcommand{\SpecialStringTok}[1]{\textcolor[rgb]{0.31,0.60,0.02}{#1}}
\newcommand{\StringTok}[1]{\textcolor[rgb]{0.31,0.60,0.02}{#1}}
\newcommand{\VariableTok}[1]{\textcolor[rgb]{0.00,0.00,0.00}{#1}}
\newcommand{\VerbatimStringTok}[1]{\textcolor[rgb]{0.31,0.60,0.02}{#1}}
\newcommand{\WarningTok}[1]{\textcolor[rgb]{0.56,0.35,0.01}{\textbf{\textit{#1}}}}
\usepackage{graphicx,grffile}
\makeatletter
\def\maxwidth{\ifdim\Gin@nat@width>\linewidth\linewidth\else\Gin@nat@width\fi}
\def\maxheight{\ifdim\Gin@nat@height>\textheight\textheight\else\Gin@nat@height\fi}
\makeatother
% Scale images if necessary, so that they will not overflow the page
% margins by default, and it is still possible to overwrite the defaults
% using explicit options in \includegraphics[width, height, ...]{}
\setkeys{Gin}{width=\maxwidth,height=\maxheight,keepaspectratio}
\IfFileExists{parskip.sty}{%
\usepackage{parskip}
}{% else
\setlength{\parindent}{0pt}
\setlength{\parskip}{6pt plus 2pt minus 1pt}
}
\setlength{\emergencystretch}{3em}  % prevent overfull lines
\providecommand{\tightlist}{%
  \setlength{\itemsep}{0pt}\setlength{\parskip}{0pt}}
\setcounter{secnumdepth}{0}
% Redefines (sub)paragraphs to behave more like sections
\ifx\paragraph\undefined\else
\let\oldparagraph\paragraph
\renewcommand{\paragraph}[1]{\oldparagraph{#1}\mbox{}}
\fi
\ifx\subparagraph\undefined\else
\let\oldsubparagraph\subparagraph
\renewcommand{\subparagraph}[1]{\oldsubparagraph{#1}\mbox{}}
\fi

%%% Use protect on footnotes to avoid problems with footnotes in titles
\let\rmarkdownfootnote\footnote%
\def\footnote{\protect\rmarkdownfootnote}

%%% Change title format to be more compact
\usepackage{titling}

% Create subtitle command for use in maketitle
\providecommand{\subtitle}[1]{
  \posttitle{
    \begin{center}\large#1\end{center}
    }
}

\setlength{\droptitle}{-2em}

  \title{homework3}
    \pretitle{\vspace{\droptitle}\centering\huge}
  \posttitle{\par}
    \author{}
    \preauthor{}\postauthor{}
    \date{}
    \predate{}\postdate{}
  

\begin{document}
\maketitle

\begin{Shaded}
\begin{Highlighting}[]
\KeywordTok{library}\NormalTok{(IsoformSwitchAnalyzeR)}
\end{Highlighting}
\end{Shaded}

\begin{verbatim}
## Loading required package: limma
\end{verbatim}

\begin{verbatim}
## Loading required package: DEXSeq
\end{verbatim}

\begin{verbatim}
## Loading required package: BiocParallel
\end{verbatim}

\begin{verbatim}
## Loading required package: Biobase
\end{verbatim}

\begin{verbatim}
## Loading required package: BiocGenerics
\end{verbatim}

\begin{verbatim}
## Loading required package: parallel
\end{verbatim}

\begin{verbatim}
## 
## Attaching package: 'BiocGenerics'
\end{verbatim}

\begin{verbatim}
## The following objects are masked from 'package:parallel':
## 
##     clusterApply, clusterApplyLB, clusterCall, clusterEvalQ,
##     clusterExport, clusterMap, parApply, parCapply, parLapply,
##     parLapplyLB, parRapply, parSapply, parSapplyLB
\end{verbatim}

\begin{verbatim}
## The following object is masked from 'package:limma':
## 
##     plotMA
\end{verbatim}

\begin{verbatim}
## The following objects are masked from 'package:stats':
## 
##     IQR, mad, sd, var, xtabs
\end{verbatim}

\begin{verbatim}
## The following objects are masked from 'package:base':
## 
##     anyDuplicated, append, as.data.frame, basename, cbind,
##     colnames, dirname, do.call, duplicated, eval, evalq, Filter,
##     Find, get, grep, grepl, intersect, is.unsorted, lapply, Map,
##     mapply, match, mget, order, paste, pmax, pmax.int, pmin,
##     pmin.int, Position, rank, rbind, Reduce, rownames, sapply,
##     setdiff, sort, table, tapply, union, unique, unsplit, which,
##     which.max, which.min
\end{verbatim}

\begin{verbatim}
## Welcome to Bioconductor
## 
##     Vignettes contain introductory material; view with
##     'browseVignettes()'. To cite Bioconductor, see
##     'citation("Biobase")', and for packages 'citation("pkgname")'.
\end{verbatim}

\begin{verbatim}
## Loading required package: SummarizedExperiment
\end{verbatim}

\begin{verbatim}
## Loading required package: GenomicRanges
\end{verbatim}

\begin{verbatim}
## Loading required package: stats4
\end{verbatim}

\begin{verbatim}
## Loading required package: S4Vectors
\end{verbatim}

\begin{verbatim}
## 
## Attaching package: 'S4Vectors'
\end{verbatim}

\begin{verbatim}
## The following object is masked from 'package:base':
## 
##     expand.grid
\end{verbatim}

\begin{verbatim}
## Loading required package: IRanges
\end{verbatim}

\begin{verbatim}
## Loading required package: GenomeInfoDb
\end{verbatim}

\begin{verbatim}
## Loading required package: DelayedArray
\end{verbatim}

\begin{verbatim}
## Loading required package: matrixStats
\end{verbatim}

\begin{verbatim}
## 
## Attaching package: 'matrixStats'
\end{verbatim}

\begin{verbatim}
## The following objects are masked from 'package:Biobase':
## 
##     anyMissing, rowMedians
\end{verbatim}

\begin{verbatim}
## 
## Attaching package: 'DelayedArray'
\end{verbatim}

\begin{verbatim}
## The following objects are masked from 'package:matrixStats':
## 
##     colMaxs, colMins, colRanges, rowMaxs, rowMins, rowRanges
\end{verbatim}

\begin{verbatim}
## The following objects are masked from 'package:base':
## 
##     aperm, apply, rowsum
\end{verbatim}

\begin{verbatim}
## Loading required package: DESeq2
\end{verbatim}

\begin{verbatim}
## Registered S3 methods overwritten by 'ggplot2':
##   method         from 
##   [.quosures     rlang
##   c.quosures     rlang
##   print.quosures rlang
\end{verbatim}

\begin{verbatim}
## Loading required package: AnnotationDbi
\end{verbatim}

\begin{verbatim}
## Loading required package: RColorBrewer
\end{verbatim}

\begin{verbatim}
## Loading required package: ggplot2
\end{verbatim}

\begin{Shaded}
\begin{Highlighting}[]
\KeywordTok{library}\NormalTok{(tidyverse)}
\end{Highlighting}
\end{Shaded}

\begin{verbatim}
## -- Attaching packages -------------------------------------------------------- tidyverse 1.2.1 --
\end{verbatim}

\begin{verbatim}
## v tibble  2.1.1     v purrr   0.3.2
## v tidyr   0.8.3     v dplyr   0.8.1
## v readr   1.3.1     v stringr 1.4.0
## v tibble  2.1.1     v forcats 0.4.0
\end{verbatim}

\begin{verbatim}
## -- Conflicts ----------------------------------------------------------- tidyverse_conflicts() --
## x dplyr::collapse()   masks IRanges::collapse()
## x dplyr::combine()    masks Biobase::combine(), BiocGenerics::combine()
## x dplyr::count()      masks matrixStats::count()
## x dplyr::desc()       masks IRanges::desc()
## x tidyr::expand()     masks S4Vectors::expand()
## x dplyr::filter()     masks stats::filter()
## x dplyr::first()      masks S4Vectors::first()
## x dplyr::lag()        masks stats::lag()
## x ggplot2::Position() masks BiocGenerics::Position(), base::Position()
## x purrr::reduce()     masks GenomicRanges::reduce(), IRanges::reduce()
## x dplyr::rename()     masks S4Vectors::rename()
## x dplyr::select()     masks AnnotationDbi::select()
## x purrr::simplify()   masks DelayedArray::simplify()
## x dplyr::slice()      masks IRanges::slice()
\end{verbatim}

\begin{Shaded}
\begin{Highlighting}[]
\KeywordTok{library}\NormalTok{(pheatmap)}
\end{Highlighting}
\end{Shaded}

\hypertarget{question-1.1}{%
\subsubsection{Question 1.1}\label{question-1.1}}

\begin{Shaded}
\begin{Highlighting}[]
\KeywordTok{setwd}\NormalTok{(}\StringTok{"/home/nuttapong/Desktop/block4/hightp/homework3/HW3_combined_handout/"}\NormalTok{)}
\NormalTok{wt1_quant <-}\StringTok{ }\KeywordTok{read_tsv}\NormalTok{(}\StringTok{"./HW3_combined_handout/salmon_result_part1/salmon_result_part1/WT1/quant.sf"}\NormalTok{)}
\end{Highlighting}
\end{Shaded}

\begin{verbatim}
## Parsed with column specification:
## cols(
##   Name = col_character(),
##   Length = col_double(),
##   EffectiveLength = col_double(),
##   TPM = col_double(),
##   NumReads = col_double()
## )
\end{verbatim}

\begin{Shaded}
\begin{Highlighting}[]
\NormalTok{wt1_quant }\OperatorTok\StringTok{ }\KeywordTok{ggplot}\NormalTok{(}\KeywordTok{aes}\NormalTok{(}\DataTypeTok{x=}\NormalTok{Length, }\DataTypeTok{y=}\NormalTok{EffectiveLength)) }\OperatorTok{+}\StringTok{ }
\StringTok{  }\KeywordTok{scale_x_continuous}\NormalTok{(}\DataTypeTok{trans=}\StringTok{'log10'}\NormalTok{) }\OperatorTok{+}\StringTok{ }\KeywordTok{scale_y_continuous}\NormalTok{(}\DataTypeTok{trans=}\StringTok{'log10'}\NormalTok{)  }\OperatorTok{+}\StringTok{ }
\StringTok{  }\KeywordTok{geom_point}\NormalTok{(}\DataTypeTok{alpha=}\FloatTok{0.3}\NormalTok{) }\OperatorTok{+}
\StringTok{  }\KeywordTok{geom_smooth}\NormalTok{() }\OperatorTok{+}\StringTok{ }\KeywordTok{geom_abline}\NormalTok{(}\DataTypeTok{color =} \StringTok{"red"}\NormalTok{, }\DataTypeTok{linetype=}\DecValTok{2}\NormalTok{)}
\end{Highlighting}
\end{Shaded}

\begin{verbatim}
## `geom_smooth()` using method = 'gam' and formula 'y ~ s(x, bs = "cs")'
\end{verbatim}

\includegraphics{homework3_files/figure-latex/unnamed-chunk-2-1.pdf}

\hypertarget{question-1.2}{%
\subsubsection{Question 1.2}\label{question-1.2}}

\hypertarget{question-1.3}{%
\subsubsection{Question 1.3}\label{question-1.3}}

\begin{Shaded}
\begin{Highlighting}[]
\KeywordTok{setwd}\NormalTok{(}\StringTok{"/home/nuttapong/Desktop/block4/hightp/homework3/HW3_combined_handout/"}\NormalTok{)}
\NormalTok{all_salmons <-}\StringTok{ }\KeywordTok{importIsoformExpression}\NormalTok{(}\DataTypeTok{parentDir =} \StringTok{"./HW3_combined_handout/salmon_result_part1/salmon_result_part1/"}\NormalTok{)}
\end{Highlighting}
\end{Shaded}

\begin{verbatim}
## Step 1 of 3: Identifying which algorithm was used...
\end{verbatim}

\begin{verbatim}
##     The quantification algorithm used was: Salmon
\end{verbatim}

\begin{verbatim}
##     Found 6 quantification file(s) of interest
\end{verbatim}

\begin{verbatim}
## Step 2 of 3: Reading data...
\end{verbatim}

\begin{verbatim}
## reading in files with read_tsv
\end{verbatim}

\begin{verbatim}
## 1 2 3 4 5 6 
## Step 3 of 3: Normalizing FPKM/TxPM values via edgeR...
## Done
\end{verbatim}

\begin{Shaded}
\begin{Highlighting}[]
\NormalTok{salmon_matrix <-}\StringTok{ }\KeywordTok{as.matrix}\NormalTok{(all_salmons}\OperatorTok{$}\NormalTok{abundance[,}\DecValTok{2}\OperatorTok{:}\KeywordTok{ncol}\NormalTok{(all_salmons}\OperatorTok{$}\NormalTok{abundance)])}
\KeywordTok{rownames}\NormalTok{(salmon_matrix) <-}\StringTok{ }\NormalTok{all_salmons}\OperatorTok{$}\NormalTok{abundance[,}\DecValTok{1}\NormalTok{]}

\NormalTok{transformed_salmon <-}\StringTok{ }\KeywordTok{log2}\NormalTok{(salmon_matrix}\OperatorTok{+}\DecValTok{1}\NormalTok{)}

\NormalTok{transformed_salmon[}\DecValTok{1}\OperatorTok{:}\DecValTok{4}\NormalTok{,]}
\end{Highlighting}
\end{Shaded}

\begin{verbatim}
##                      WT1       WT2       WT3    WTTPA1    WTTPA2 WTTPA3
## TCONS_00000001 0.2973299 0.0000000 0.0000000 0.3822156 0.0000000      0
## TCONS_00000002 0.0000000 0.0000000 0.0000000 0.0000000 0.0000000      0
## TCONS_00000003 0.0000000 0.2984888 0.2253968 1.0124265 0.0000000      0
## TCONS_00003946 0.0392366 0.0000000 0.1913649 0.0000000 0.0564598      0
\end{verbatim}

\hypertarget{question-1.4}{%
\subsubsection{Question 1.4}\label{question-1.4}}

\begin{Shaded}
\begin{Highlighting}[]
\NormalTok{salmon_tibble <-}\StringTok{ }\KeywordTok{as_tibble}\NormalTok{(transformed_salmon, }\DataTypeTok{rownames=}\OtherTok{NA}\NormalTok{)}

\NormalTok{top100var <-}\StringTok{ }\NormalTok{salmon_tibble }\OperatorTok\StringTok{ }\KeywordTok{rownames_to_column}\NormalTok{() }\OperatorTok\StringTok{ }\KeywordTok{rowwise}\NormalTok{() }\OperatorTok\StringTok{ }
\StringTok{  }\KeywordTok{mutate}\NormalTok{(}\DataTypeTok{variance=}\KeywordTok{var}\NormalTok{(}\KeywordTok{c}\NormalTok{(WT1, WT2, WT3, WTTPA1, WTTPA2, WTTPA3))) }\OperatorTok\StringTok{ }
\StringTok{  }\KeywordTok{arrange}\NormalTok{(}\KeywordTok{desc}\NormalTok{(variance)) }\OperatorTok\StringTok{ }\KeywordTok{slice}\NormalTok{(}\DecValTok{1}\OperatorTok{:}\DecValTok{100}\NormalTok{)}
\end{Highlighting}
\end{Shaded}

\hypertarget{question-1.5}{%
\subsubsection{Question 1.5}\label{question-1.5}}

\begin{Shaded}
\begin{Highlighting}[]
\NormalTok{top100var_mat <-}\StringTok{ }\KeywordTok{as.matrix}\NormalTok{(top100var[,}\DecValTok{2}\OperatorTok{:}\DecValTok{7}\NormalTok{])}
\KeywordTok{rownames}\NormalTok{(top100var_mat) <-}\StringTok{ }\KeywordTok{as.data.frame}\NormalTok{(top100var)[,}\DecValTok{1}\NormalTok{]}
\KeywordTok{pheatmap}\NormalTok{(top100var_mat, }\DataTypeTok{show_rownames =} \OtherTok{TRUE}\NormalTok{, }\DataTypeTok{cellheight =} \DecValTok{10}\NormalTok{, }\DataTypeTok{height =} \DecValTok{10}\NormalTok{)}
\end{Highlighting}
\end{Shaded}

\includegraphics{homework3_files/figure-latex/unnamed-chunk-6-1.pdf}


\end{document}
